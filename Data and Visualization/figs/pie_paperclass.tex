
%\subsection{Pie chart for Paper class (197)}
%\begin{figure}
\begin{center}
\begin{tikzpicture}[scale=2]
\pgfmathsetcounter{pieb}{0}
\foreach \p/\q/\t/\c in {40/78/Evaluation Research/blue!20, 35/69/Proposal of Solution/blue!30, 25/49/Validation Research/blue!40, 1/1/Philosophical Papers/blue!50}
  {
    \setcounter{piea}{\value{pieb}}
    \addtocounter{pieb}{\q}
    \slice{\thepiea/197*360}
          {\thepieb/197*360}
          {\p\%}{\t}{\c}
  }
\end{tikzpicture}
\textbf{Pie chart for Paper class (197)}
\end{center}
%\caption{Pie chart for Paper class (197)}
%\label{fig:pie_paperclass}
%\end{figure}

