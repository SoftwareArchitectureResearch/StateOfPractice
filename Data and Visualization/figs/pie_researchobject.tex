
%\subsection{Pie chart for Research Object (170)}
%\begin{figure}
\begin{center}
\begin{tikzpicture}[scale=2]
\pgfmathsetcounter{pieb}{0}
\foreach \p/\q/\t/\c in {18/31/Architecture Analysis Method/blue!20, 15/26/Architecture Design Method/blue!30, 11/18/Architecture Decision Making/blue!40, 10/17/Architecture Optimization Method/blue!50, 9/16/Reference Architecture/blue!60, 7/12/Architectural Aspects/blue!70, 6/10/Architecture Description Language/blue!80, 6/11/Architecture Pattern/blue!90, 5/9/Architecture Evolution/blue!100, 4/7/Architecture Description/blue!110, 4/7/Architecture Extraction/blue!120, 2/3/Technical Debt/blue!130, 2/3/Architectural Assumptions/blue!140}
  {
    \setcounter{piea}{\value{pieb}}
    \addtocounter{pieb}{\q}
    \slice{\thepiea/170*360}
          {\thepieb/170*360}
          {\p\%}{\t}{\c}
  }
\end{tikzpicture}
\textbf{Pie chart for Research Object (170)}
\end{center}
%\caption{Pie chart for Research Object (170)}
%\label{fig:pie_researchobject}
%\end{figure}

